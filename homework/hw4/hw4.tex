\documentclass[11pt]{article}

\usepackage{fullpage,amsmath,amssymb,enumitem}

\begin{document}


{\Large{\noindent Databases Homework 4}}

\begin{enumerate}
	\item Below are two relations.  Each relation has a corresponding set of FDs that
	hold in the relation.
	For each relation and corresponding  FDs, answer these three questions:
	
	\begin{enumerate}[label=(\roman*)]
		\item What are all the nontrivial FDs that follow from the given FDs?  List each FD with
		a single attribute on the right side.
		\item What are the keys of the relation?
		\item What are the superkeys for the relation that are not keys?
	\end{enumerate}
	
	Schemas and FDs:
	
	\begin{enumerate}
		\item $R(A, B, C, D)$ with FDs $C \to B$, $C \to A$, and $AB \to D$.
		\item $R(A, B, C, D)$ with FDs $AB \to C$, $C \to D$, and $D \to B$.
		%\item $R(A, B, C, D, E, F)$ with FDs $E \to F$, $E \to D$, $AD \to B$, $AD \to C$, and $F \to B$.

	\end{enumerate}
	
	\item For all the parts of this question, only write FDs which are completely nontrivial and which have single
	attributes on the right side.  Also, in each question, you only need to provide a minimal basis of FDs (though if you
	give extras, that's OK).
	
	\begin{enumerate}
		\item Suppose we have a relation PizzaOrders(customerId, customerName, pizzaId, pizzaName, orderTime, quantity, 
		slices) which has the following FDs:
		
		customerId $\to$ customerName \\
		pizzaId $\to$ pizzaName \\
		customerId orderTime pizzaId  $\to$ quantity slices
		
		Project these FDs onto the relation $X$(customerId, customerName, quantity, orderTime).
		
		\emph{In other words, suppose relation $X = \pi_\text{customerId, customerName, quantity, orderTime}(\text{PizzaOrders})$. 
		Determine a minimal basis of FDs that hold in $X$}.
		
		\item Suppose we have a relation TakeCourses(studentId, studentName, profId, profName, courseNumber, department, capacity)
		which has the following FDs:
		
		studentId $\to$ studentName \\
		profId $\to$ profName \\
		courseNumber department $\to$ profId capacity
		
		Project these FDs onto the relation $Y$(profName, studentName, courseNumber, department).
		
		\item Suppose we have a relation $R(A, B, C, D, E, F)$ which has the following FDs:
		
		$B \to A$, $E \to C$, $F \to D$, $CD \to B$, $CF \to A$.
		
		Project these FDs onto the relation $Z(A, E, F)$.
	\end{enumerate}

\item Consider the relation $R(A, B, C, D, E)$ with FDs $C\to AD$, $DE \to B$, $A \to E$,
	and $B \to C$.
	\begin{enumerate}
		\item List all the keys of $R$.
		\item Indicate which FDs are BCNF violations.
		\item Decompose R, if necessary, into relations that are in BCNF.  You can choose
		any BCNF violations you want to do the decomposition.  Show your work.
		\item Is the result of your decomposition dependency-preserving?  In other words,
		is the set of FDs that hold in the individual relations in your decomposition equivalent
		to the original set of FDs in R?  Explain your answer.  If your decomposition is
		not dependency-preserving, re-decompose R into 3NF.  Show your work.
		
	\end{enumerate}
	
	\item Consider a relation Stocks$(B, O, I, S, Q, D)$, whose attributes may be thought of informally as broker, office (of the broker), investor, stock, quantity (of the stock owned by the investor), and dividend (of the stock). Let the set of FDs for Stocks be $S \to D$, 
	$I \to B$, $IS \to Q$, and $B \to O$.
	
	\begin{enumerate}
		\item List all the keys for Stocks.
		\item Use the 3NF synthesis algorithm to find a lossless-join, dependency-preserving decomposition of Stocks into a set of 3NF relations. Show your work.
		\item Are any of the relations in your final decomposition not in BCNF?  If yes, decompose them into BCNF.
	\end{enumerate}
	
	
\end{enumerate}

\end{document}
